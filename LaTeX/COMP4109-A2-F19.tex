% !TEX program = pdfLaTeX
 
\documentclass[12pt, letterpaper]{article}

 
\usepackage{amssymb,amsmath}       % for more math
\usepackage{fancyvrb}              % for fancy verbatim mode
\usepackage{xcolor}                % to use colours
\usepackage[margin=2cm]{geometry}  % page layout
\usepackage{graphicx}              % to include grahics
\usepackage{array}        
\usepackage{pifont}                % for some fancy characters 
\usepackage{listings}
\usepackage{color}
\usepackage[latin1]{inputenc}

\definecolor{dkgreen}{rgb}{0,0.6,0}
\definecolor{gray}{rgb}{0.5,0.5,0.5}
\definecolor{mauve}{rgb}{0.58,0,0.82}

\lstset{frame=tb,
  language=Python,
  aboveskip=3mm,
  belowskip=3mm,
  showstringspaces=false,
  columns=flexible,
  basicstyle={\small\ttfamily},
  numbers=none,
  numberstyle=\tiny\color{gray},
  keywordstyle=\color{blue},
  commentstyle=\color{dkgreen},
  stringstyle=\color{mauve},
  breaklines=true,
  breakatwhitespace=true,
  tabsize=3
}

\usepackage[pdftitle={COMP4109 Assignment},
            pdfsubject={Carleton University, 
                        COMP4109, Fall 2019},
            pdfauthor={M. Jason Hinek}]%
            {hyperref}
            
            
%%
%% finding last page number if needed \ref{LastPage}
%%
\usepackage{lastpage}              % to find the last page


%%
%% set some fancy headers/footers
%%
\usepackage{fancyhdr}              % fancy headers
\pagestyle{fancy}
\lhead{\large \textbf{COMP4109 - Assignment 2}}
\rhead{\large {\textbf{Due \textcolor{red}{Wednesday, Nov 20, 10:25am}}}}
\lfoot{Applied Cryptography}
\cfoot{}
\rfoot{\thepage\ of \pageref{LastPage}}

%%
%% some things to make my life easier
%%

%% bold green text
\newcommand{\bgreen}[1]{\textbf{\textcolor{green!50!black}{#1}}}

%% the set of integers Z
\newcommand{\ZZ}{\ensuremath{\mathbb{Z}}}

%% writing typewriter style (for code)
\newcommand{\code}[1]{\texttt{#1}}

 
\newcommand{\answer}[1]{{\small{\color{blue}#1}}}
\renewcommand{\answer}[1]{}

% I don't like paragraphs indented   
\setlength\parindent{0pt}
 
% --------------------------------------------------------------------------- %   
\begin{document}

{\Large
\begin{tabular}{rm{6cm}rm{6cm}}
Name :  & Lachlan Campbell & Name : & Jonathan Vidal           \\
ID :    & 100999056   & ID :   & 101002247
\end{tabular}
}

\bigskip
\hrule
\bigskip


You can work in groups of two for the assignments. Each group will submit a single
typeset hardcopy of your solution to be submitted to cuLEarn. Be sure to include the name and student id for
each person working on the assignment.  

\bigskip   % a big skip (horizontal spacing)
\hrule     % line across the page
\bigskip   % a big skip (horizontal spacing)
\hrule     % line across the page

\section{Hash Functions}

Let $f$, $g$ and $h$ be hash functions that each map binary strings of length $2n$ to binary strings of length $n$. That is,
\[ f\colon \{0,1\}^{2n} \to \{0,1\}^n \]
\[ g\colon \{0,1\}^{2n} \to \{0,1\}^n \]
\[ h\colon \{0,1\}^{2n} \to \{0,1\}^n \]

Suppose that  

 \[h(x) = f( g(x) || g(x) ),\]

where $a||b$ means $a$ concatenated with $b$.  
Let $CR$ be the set of all collision resistant hash functions. Prove that if $f \in CR$ and $g \in CR$ then $h \in CR$. That is, prove that $f \in CR \land g \in CR \Rightarrow h \in CR$.

\subsection{Proof by Contrapositive}
$h(x) \notin CR \Rightarrow f(x) \notin CR \lor g(x) \notin CR$
~\\~\\
If we assume $h$ is not collision resistant, we can find two elements $x, y \in \{0, 1\}^{2n}$ such that $x \neq y$ and $h(x) = h(y)$. Assuming $h$ is not collision resistant, this task should be relatively easy. 
~\\~\\
We have two cases to consider.
~\\~\\
\textbf{Case 1:}
In this case we assume $g$ is not collision resistant. If it is not then $g(x) = g(y)$ for some elements $x, y \in \{0, 1\}^{2n}$ where $x \neq y$. If this is the case then it does not matter whether or not $f$ is collision resistant. Since $g(x)$ produces the same hash as $g(y)$, $g(x)||g(x) = g(y)||g(y)$ so the function $f$ will also produce the same hash regardless of its quality. 
~\\~\\
\textbf{Case 2:}
In this case we assume $g$ \textit{is} collision resistant. If this is the case then we have $g(x) \neq g(y)$ for some elements $x, y \in \{0, 1\}^{2n}$ where $x \neq y$. If this is the case then we must have $f(g(x)||g(x)) = f(g(y)||g(y))$ in order for us to have $h \notin CR$. Therefore $f$ must not be collision resistant.
~\\~\\
In \textbf{Case 1} and \textbf{Case 2}, we have shown that $g$ and $f$ cannot both be in $CR$ if $h \notin CR$. Therefore, by the nature of the contrapositive, we have proven that if $f \in CR \land g \in CR \Rightarrow h \in CR$.

\newpage

\section{Textbook RSA}

In each of these problems, consider textbook RSA.  
The modulus $N = pq$ is the product of two primes $p$ and $q$ and the public/private exponents $e/d$ are computed
as inverses modulo $\phi(N)$.  That is, $ed \equiv 1 \bmod {\phi(N)}$, where $\phi(N) = (p-1)(q-1)$.


\begin{itemize}

\item[a)] Consider the RSA public key 
$(e,N) = (9292162750094637473537, 13029506445953503759481)$.
Find the corresponding private exponent $d \in \mathbb{Z}_N^*$.  
Show how you did this. 

\subsection{Determining $d$ through Wiener's Attack}
We will use Weiner's attack to find the private exponent $d$ from the given $e$ and $N$. First we will determine all of the convergents of $\frac{e}{N}$. From each convergent, we will determine if one will solve the equation $f(x)=x^{2} - ((N-\phi(N)+1)x+ N = 0$. Solving this equation will tell us if there exists two integer roots that will make the equation
$1 \equiv ed \bmod \phi(N)$ true. If we find the convergent $\frac{k}{d}$ that makes the statement true, then we found our $k$: the denominator. The convergent that solves $f(x)$ is $\frac{363}{509}$, $\phi(N) = 13029506445724987531764$. We plug them into our equation to see if the equation holds:
\begin{equation*}
	\begin{split}
		1 & \equiv ed \bmod \phi(N)\\
		1 & \equiv (9292162750094637473537)(509) \bmod 13029506445724987531764\\
		1 & \equiv 4729710839798170474030333 \bmod 13029506445724987531764\\
		1 & \equiv 1
	\end{split}
\end{equation*}
$\therefore$ our private exponent $d = 509$
\subsection{Code used for computation}
Python code used for Weiner's attack is below. This is from GitHub repositories \href{https://github.com/orisano/owiener}{owiener} and \href{https://github.com/pablocelayes/rsa-wiener-attack}{rsa-wiener-attack}.
\begin{lstlisting}
from math import isqrt

def generate_continued_fractions(n, d):
    l = []
    while d != 0:
        i = n // d
        n, d = d, n - i*d
        l.append(i)
    return l

def determine_convergents(fracs):
    l = []
    n0, d0 = 0, 1
    n1, d1 = 1, 0
    for q in fracs:
        n = q * n1 + n0
        d = q * d1 + d0
        n0, d0 = n1, d1
        n1, d1 = n, d
        l.append((n,d))
    return l

def wieners_attack(e, N):

    fracs = generate_continued_fractions(e,N)
    convergents = determine_convergents(fracs)

    for k,d in convergents:
        #prevent division by 0
        if k == 0:
            continue

        phi = (e*d-1) // k
        b = N - phi + 1

        #check for integer roots of equation x^2 - ((N - phi) + 1)x + N = 0
        discr = b*b - 4*N
        if discr < 0:
            continue

        if is_perfect_square(discr) and (b+discr) % 2==0:
            return d

def is_perfect_square(num):
    root = isqrt(num)
    return root*root == num

e = 9292162750094637473537
N = 13029506445953503759481

d = wieners_attack(e,N)
print(d)
\end{lstlisting}

\item[b)] Using the RSA public key $(e,N)$, compute the ciphertext for the plaintext message $m$ provided in 
file \bgreen{rsa.txt}.
Provide your code or a transcript of your computation.

\subsection{RSA encryption}
The encryption rule for RSA is given by the formula $m^{e}\bmod N$. Since the exponent is small enough, this can be calculated quickly using the
given formula in a Python script. The ciphertext for the plaintext message $m$ provided in the file is 19700101.\newpage
\subsection{Code used for computation}
\begin{lstlisting}
e = #e value found from rsa.txt
N = #N value found from rsa.txt
m = #m value found from rsa.txt

c = (m ** e) % N
print(c)
\end{lstlisting}

\item[c)] Using the RSA public key $(e,N)$, compute the ciphertext for the plaintext message $m$ provided in 
file \bgreen{rsa.txt}.
Provide your code (if it is different than that used in part b) 
or a transcript of your computation.

\subsection{RSA encryption through optimized algorithm}
The encryption rule for RSA is given by the formula $m^{e}\bmod N$. Because the exponent is quite large, calculating the ciphertext through the encryption rule
is infeasible due to the large calculations required. The algorithm will have to be optimized. We will build the exponent $e$ in binary to reduce the number of multiplications
required to compute the ciphertext. Our final ciphertext is 1223334444555556666667777777888888889999999990000000000.
\subsection{Code used for computation}
\begin{lstlisting}
e = #e value found from rsa.txt
N = #N value found from rsa.txt
m = #m value found from rsa.txt

def optimized_RSA(a, b, N):
    bits = [int(i) for i in list(bin(b))[2:]]
    S = 1
    n = len(bits)
    for i in bits:
        S = S*S % N
        if i == 1:
            S = S*a % N
    return S

c = optimized_RSA(m,e,N)
print(c)
\end{lstlisting}

\end{itemize}

\newpage

\section{RSA-OAEP}

Consider RSA-OAP as described in the Chapter 10 of the textbook and detailed in RCF 8017 (also specified as PKCS \# 1). In particular, look at RSAES-OAEP. 

\medskip

\url{https://tools.ietf.org/html/rfc8017}

\medskip

In OAEP, we need two hash functions that can output variable length hashes. However, most hash functions have a fixed output length. In order to achieve variable length hashes, we use a \emph{Mask Generation Function}. In particular this standard specifies that MGF1 be used. 

\medskip
To simplify the illustration of MGF1, instead of using an underlying hash function like SHA-1 or SHA-256, suppose the underlying hash function is given by

\[ h(x) = 2x + 1  \pmod {65521}. \] 

Note that $65521 < 65536 = 2^{16}$ and so the output can always be considered as occupying two octets (bytes). You will assume that the output occupies two octets (bytes). Using this hash function, compute the output of 

\[  \textrm{MGF1}(2^7+2^9+2^{10}, 5) \]

That is, what are the 5 octets (bytes) of output from this call to the function. 

\subsection{Output}
\begin{center}
	\begin{tabular}{|r|l|}
		\hline
		\textbf{Octet Number}	&\textbf{Octet Value} 	\\ \hline
		1	&0x00	\\
		2	&0x00m	\\
		3	&0xa6	\\
		4	&0x00	\\
		5	&0x00	\\
		\hline
	\end{tabular}
\end{center}

\subsection{Comments on results}

The code (pictured below) was adapted from the MGF1 python implementation on the \href{https://en.wikipedia.org/wiki/Mask\_generation\_function\#Example\_Code}{MGF1 Wikipedia page}. In order to make it work with the supplied hash function I had to make some changes to the i2osp function (as pictured below) as well as make some changes to how the input and output of the MGF1 function and the h function was formatted (using struct.pack and struct.unpack). I was using big-endian formatting. I'm not too confident with the results but I at the very least, got five octets of output, so there must be something there.

\newpage
\subsection{Code used for computation}
\begin{lstlisting}
import struct

def h(x):
    y = struct.unpack(">q", x)
    return struct.pack(">I", ((2*y[0] + 1) % 65521))

def i2osp(integer, size=4):
    return struct.pack(">I", integer)

def mgf1(input, length):
    counter = 0
    output = b''
    
    while (len(output) < length):
        C = i2osp(counter, 4)
        output += h(input + C)
        counter += 1
  
    return output[:length+1]
    
print(mgf1(struct.pack(">I", ((2**7) + (2**9) + (2**10))), 5))
\end{lstlisting}

\newpage

\section{ElGamal Signatures}

Consider the ElGamal signature scheme, as described in the 
lecture notes, where the hash function is the identity 
function (i.e., $H(m) = m$).

\begin{itemize}
\item[a)] Consider an instance with public key 
$pk = (p,g,y) = (65537, 3, {20031})$.
Suppose you are given the following two message-signature
pairs 
\begin{align*}
\langle m_1, \sigma_1 \rangle = \langle 31415, (40071, 17134)\rangle \\
\langle m_2, \sigma_2 \rangle = \langle 32768, (40071, 22525)\rangle
\end{align*}
Notice that the value of $r=g^k$ is the same for both.  (That is, both signatures were 
generated with the same $k$ value. First, outline a method (algorithm) to 
recover both $k$ and the private key $x$ in this situation. 
Provide the equations needed to solve for $k$ and $x$.  Then, 
compute the value of $k$ used and compute the private key $x$
using this method.  

\subsection{Solve for $k$}
We first isolate $k$ given the ElGamal signatures by avoiding creating a discrete log problem. We introduce $\sigma_1$ and $\sigma_2$ into a single equation.
\begin{equation*}
	\begin{split}
		s_1 - s_2 & \equiv k^{-1}((H(m_1)- xr_1) - k^{-1}(H(m_2)- xr_2) \bmod p-1 \\
		& \equiv k^{-1}((H(m_1)- xr_1) - (H(m_2)- xr_2)) \bmod p-1 \\
		& \equiv k^{-1}(H(m_1)- xr_1 - H(m_2)+ xr_2) \bmod p-1 \\
	\end{split}
\end{equation*}
Because $r_1 = r_2 = 40071$, we can replace $r_2$ with $r_1$ and simplify further.
\begin{equation*}
	\begin{split}
		s_1 - s_2 & \equiv k^{-1}(H(m_1)- xr_1 - H(m_2)+ xr_1) \bmod p-1 \\
		s_1 - s_2 & \equiv k^{-1}(H(m_1)- H(m_2)) \bmod p-1 \\
		k (s_1 - s_2) & \equiv (H(m_1)- H(m_2)) \bmod p-1 \\
		k & \equiv (H(m_1)- H(m_2)) * (s_1 - s_2)^{-1} \bmod p-1 \\
		k & \equiv (m_1- m_2) * (s_1 - s_2)^{-1} \bmod p-1 \\
	\end{split}
\end{equation*}
We have created an equation for $k$. To solve for k, we plug in $m_1$, $m_2$, $s_1$, $s_2$, and $p$ into the equation:
\begin{equation*}
	\begin{split}
		k & \equiv (m_1- m_2) * (s_1 - s_2)^{-1} \bmod p-1 \\
		k & \equiv (32768-31415) * (22525-17134)^{-1} \bmod 65537-1 \\
		k & \equiv 1353 * 5391^{-1} \bmod 65536 \\
		k & \equiv 1353 * 14831 \bmod 65536 \\
		k & \equiv 12327 \\
	\end{split}
\end{equation*}
We found the inverse of 5391 using the \href{https://en.wikibooks.org/wiki/Algorithm_Implementation/Mathematics/Extended_Euclidean_algorithm#Iterative_algorithm_3}{Extended Euclidean Algorithm}.\newline \\ $\therefore$ our $k$ that produced the two ElGamal signatures is 12327.

\subsection{Solve for $x$}
Using the equation $H(m)\equiv sk + rx \bmod p-1$, we can isolate $x$.
\begin{equation*}
	\begin{split}
		H(m) & \equiv sk + rx \bmod p-1 \\
		H(m) - sk & \equiv rx \bmod p-1 \\
		r^{-1}(H(m) - sk) & \equiv x \bmod p-1 \\
		x & \equiv r^{-1}(H(m) - sk) \bmod p-1 \\
		x & \equiv (40071)^{-1} (31415 - (17134)(12327)) \bmod 65537 - 1\\
		x & \equiv (40071)^{-1} (31415 - (17134)(12327)) \bmod 65537 - 1\\
		x & \equiv 11811\\
	\end{split}
\end{equation*}
We found the inverse of 40071 using the \href{https://en.wikibooks.org/wiki/Algorithm_Implementation/Mathematics/Extended_Euclidean_algorithm#Iterative_algorithm_3}{Extended Euclidean Algorithm}.\newline \\ $\therefore$ our $x$ that produced the two ElGamal signatures is 11811.

\item[b)] Consider an instance with public key $pk = (p,g,y) = (65537, 3, 25733)$.  Determine which of the following is a valid signature for the message $m = 8676$ and which is not.  Show your work (or a trace of a program you write) for each.
\begin{align*}
\sigma_1 &= (42679, 17035) \quad \\ 
\sigma_2 &= (42679, 17036) \quad
\end{align*}

\subsection{Check if signature verification properties hold}
We must first verify if the inequalities $0<r<p$ and $0<s<p-1$ hold for $\sigma_1$ and $\sigma_2$.
\begin{align*}
&0<r<p \quad \\ 
&0<42679<65537 \quad
\end{align*}
\begin{align*}
&0<s<p-1 \quad \\ 
&0<17035<65536 \quad
\end{align*}
$\therefore$ both properties hold for $\sigma_1$.
\begin{align*}
&0<r<p \quad \\ 
&0<42679<65537 \quad
\end{align*}
\begin{align*}
&0<s<p-1 \quad \\ 
&0<17036<65536 \quad
\end{align*}
$\therefore$ both properties hold for $\sigma_2$.\newline \\
As both properties hold for $\sigma_1$ and $\sigma_2$, we now must check if the equation $g^{H(m)}=y^{r}r^{s}$ holds. As $H(m)= m$, the equation can be simplifed to $g^{m}=y^{r}r^{s}$. I created a Python script that evaluated both sides of the equation to see if the property holds.
The property holds for $\sigma_2$ and not for $\sigma_1$.\newline \\ $\therefore \sigma_2$ is a valid signature for the message $m$ while $\sigma_1$ is not a valid signature for the message $m$.

\subsection{Code used for computation}
\begin{lstlisting}
m = 8676
y = 25733
p = 65537
g = 3

sig = [(42679, 17035),(42679, 17036)]

for s in sig:
    eq1 = (g**m) % p
    eq2 = ((y**s[0]) * (s[0]**s[1])) % p
    print("signature", s,":",eq1 == eq2)
\end{lstlisting}

\end{itemize}

\newpage

\section{Finding Primitive Elements in $\mathbb{Z}_p^*$}

First, explain how we determine if a given element in $\mathbb{Z}_p^*$ is primitive or not. 
Using this, then explain how primitive elements are found in practice. 

\subsection{Determining if an Element is Primitive in  $\mathbb{Z}_p^*$}
An element is primitive in  $\mathbb{Z}_p^*$ if every coprime $c$ of $n$ can be represented as the power of $g$ modulo $n$ given by this formula: $$g^k mod(n) = k$$ If an element is primitive, it also means it is a generator of $\mathbb{Z}_p^*$ (ie. if $x$ is a primitive element in  $\mathbb{Z}_p^*$ then every number in  $\mathbb{Z}_p^*$ can be represented with the function: $x^k mod(n) = k$.

\subsection{Finding Primitives in  $\mathbb{Z}_p^*$}


\end{document}
